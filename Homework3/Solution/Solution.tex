\documentclass[a4paper, 10pt]{article}

\usepackage[margin = 1in]{geometry} % for spacing around
\usepackage{graphicx} % for including images in your pdfs
\usepackage{xcolor} % for including colors in your pdf
\usepackage{soul} % for text decoration
\usepackage[utf8]{inputenc} % for encoded text
\usepackage[T1]{fontenc}
\usepackage{setspace} % for setting different line spacings between paragrafs.
\usepackage{enumerate} % for letting us get more detailed enumerate lists
\usepackage{multirow} % to let us combine more rows together
\usepackage{colortbl} % for decorating tables
\usepackage{amsmath} % used for representing more complicated math displays
\usepackage{supertabular}
\usepackage{longtable} % both of these packages are used to making really big tables
\usepackage{wrapfig} % allows us to wrap text around figures
\usepackage{fancyhdr} % for making fancy headers
%\usepackage{bibtex} % for making better bibliographies
\usepackage[pdftex]{hyperref} % for letting us make links
\usepackage{lscape} % Allows us to flip from portrait to landspace
\usepackage{tikz} % for high detailed drawing
\usepackage{multicol} % To put things side by side
\usepackage{rotating} % For rotating objects
% \usepackage{draftwatermark} % For adding watermarks
\usepackage{MnSymbol} % for using multiple symbols
\usepackage{mathtools} % Used for more math symbols
\usepackage{xfrac} % For more complciated fractions and to add derivitives
\usepackage{hyperref} % for hyper links
\usepackage{enumitem} % for better enum lists
\usepackage{tcolorbox} % for adding colored text boxes
\usepackage{bm} % Adding bold text to math inputs
\usepackage{pgfplots} % Used for plotting functions
\usepackage{background}
\usepackage{amsfonts}

% Settings for the background package
\backgroundsetup{
    scale=1, % Size - adjust as necessary
    color=black, % Color of the logo
    opacity=0.1, % Opacity - adjust to your liking; 0 = fully transparent, 1 = fully opaque
    angle=0, % Rotation angle if you want to rotate the logo
    position=current page.center, % Position - you can change this as necessary
    vshift=0cm, % Vertical shift - adjust as necessary
    hshift=0cm, % Horizontal shift - adjust as necessary
    contents={\includegraphics[width=10cm]{Logo}} % Replace 'your_logo_filename' with the name of your logo file
}

% Setting up the default image path
\graphicspath{{../../}}

% Implementing authro details
\title{Discrete Mathemathics Homework}
\author{Emre Arapcic-Uevak}
\date{}

% Setting up the fancy page style
\fancypagestyle{customStyle}{
	\lhead{} \chead{} \rhead{}
	\lfoot{} \cfoot{\thepage} \rfoot{}
	\renewcommand{\headrulewidth}{0pt}
	\renewcommand{\footrulewidth}{1pt}
}
\pagestyle{customStyle}

% Setting up hyperref options
\hypersetup {
	colorlinks = false,
	citecolor = black,
	filecolor = blue,
	linkcolor = blue,
	urlcolor = blue,
	pdftex
}

% Custom commands


%! language = Latex
\begin{document}
    \maketitle
    \vspace{5mm}

    \begin{abstract}
        \begin{center}
            \noindent This document encompasses a detailed set of solutions for discrete mathematics homework assignments. Covering core topics such as set theory, logic, relations, and graph theory, each solution is meticulously worked out to ensure clarity and accuracy. The problems addressed in this compilation have been selected based on their relevance and the challenges they pose, offering students a comprehensive resource to validate their work and deepen their understanding of the subject. By presenting both the problem statements and their corresponding solutions, this collection aims to be a valuable reference for students navigating the intricacies of discrete mathematics.
        \end{center}
    \end{abstract}
    \pagebreak

    \tableofcontents
    \pagebreak


    \section{Chapter 3}

    \subsection{Problem 1}
    \begin{enumerate}
        \item \( x = \frac{1}{2} \) is a counterexample for \( \forall x \in \mathbb{R}, x \geq \frac{1}{x} \) because \( \frac{1}{2} < 2 \).
        \item \( a = 0 \) is a counterexample for \( \forall a \in \mathbb{Z}, \frac{(a - 1)}{a} \) is not an integer because \( \frac{(1 - 1)}{1} = 0 \) is an integer.
        \item \( m = 2 \), \( n = 1 \) is a counterexample for all positive integers \( m \) and \( n \), \( m * n \geq m + n \) because \( 2 * 1 < 2 + 1 \).
        \item \( x = 1 \), \( y = 2 \) is a counterexample for all real numbers \( x \) and \( y \), \( \sqrt{x + y} = \sqrt{x} + \sqrt{y} \) because \( \sqrt{1 + 2} \neq \sqrt{1} + \sqrt{2} \).
    \end{enumerate}

    \subsection{Problem 2}
    \textbf{If... then... form:}
    \begin{enumerate}
        \item If a triangle is equilateral, then it is isosceles.
        \item If someone is a computer science student, then they need to take data structures.
    \end{enumerate}

    \noindent \textbf{Direct statement form:}
    \begin{enumerate}
        \item For any triangle, being equilateral implies being isosceles.
        \item All computer science students are required to take data structures.
    \end{enumerate}

    \subsection{Problem 3}
    \begin{itemize}
        \item Statement a is \textbf{true}. All odd numbers in \( D \) are bigger than 0
        \item Statement b is \textbf{true}. All negative numbers in \( D \) are even.
        \item Statement c is \textbf{false}. Counterexample: \( x = 16 \) is even and greater than 0.
        \item Statement d is \textbf{true}. For \( x \in D \) with ones digit 2, the tens digit is 3 or 4 (e.g., 32).
        \item Statement e is \textbf{false}. Counterexample: \( x = 36 \) has a ones digit of 6 but the tens digit is not 1 or 2.
    \end{itemize}

    \subsection{Problem 4}
    \textbf{Statement 29:}
    \begin{itemize}
        \item Original: \( \forall x \in \mathbb{R}, \exists y \in \mathbb{R} \text{ such that } x < y \) is \textbf{true}.
        \item Interchanged: \( \exists y \in \mathbb{R} \text{ such that } \forall x \in \mathbb{R}, x < y \) is \textbf{false}.
    \end{itemize}

    \noindent \textbf{Statement 30:}
    \begin{itemize}
        \item Original: \( \exists x \in \mathbb{R} \text{ such that } \forall y \in \mathbb{R^-}, x > y \) is \textbf{true}.
        \item Interchanged: \( \forall y \in \mathbb{R^-}, \exists x \in \mathbb{R} \text{ such that } x > y \) is \textbf{true}.
    \end{itemize}

    \pagebreak
    \subsection{Problem 5}
    \begin{enumerate}
        \item \textbf{Argument 7:} Commits the converse error.
        \begin{quote}
            All healthy people eat an apple a day.\\
            Keisha eats an apple a day.\\
            \emph{Therefore, Keisha is a healthy person.}
        \end{quote}

        \item \textbf{Argument 8:} Valid by universal modus ponens.
        \begin{quote}
            All freshmen must take a writing course.\\
            Caroline is a freshman.\\
            \emph{Therefore, Caroline must take a writing course.}
        \end{quote}

        \item \textbf{Argument 9:} Commits the inverse error.
        \begin{quote}
            If a graph has no edges, then it has a vertex of degree zero.\\
            This graph has at least one edge.\\
            \emph{Therefore, this graph does not have a vertex of degree zero.}
        \end{quote}

        \item \textbf{Argument 10:} Valid by universal modus tollens.
        \begin{quote}
            If a product of two numbers is 0, then at least one of the numbers is 0.\\
            For a particular number \(x\), neither \(2x + 1\) nor \(x - 7\) equals 0.\\
            \emph{Therefore, the product \((2x + 1)(x - 7)\) is not 0.}
        \end{quote}
    \end{enumerate}

    \pagebreak
    \section{Chapter 4}

        \subsection{Problem 6}

    \section*{Chapter 4 Problem 6 Justifications}
        \begin{enumerate}
            \item Assume that \( k \) is a particular integer:
            \begin{itemize}
                \item \( -17 \) is odd because it can be represented in the form \( 2n + 1 \), where \( n = -9 \).
                \item \( 0 \) is even because \( 2 \times 0 = 0 \), and any integer expressible as \( 2n \) for integer \( n \) is even.
                \item For \( k \) being any integer, \( 2k - 1 \) is odd:
                    \begin{itemize}
                        \item We know \( 2k \) is even since it can be written as \( 2 \times k \), where \( k \) is an integer.
                        \item If \( 2k - 1 \) were even, there would exist an integer \( m \) such that \( 2k - 1 = 2m \).
                        \item Adding 1 to both sides gives \( 2k = 2m + 1 \), suggesting \( 2k \) is odd, which is a contradiction.
                        \item Thus, \( 2k - 1 \) cannot be even, and since every integer is either even or odd and 2k - 1 is not even then it must be odd.
                    \end{itemize}
            \end{itemize}

            \item Assume that \( c \) is a particular integer:
            \begin{itemize}
                \item \( -6c \) is even as it can be written as \( 2 \times (-4 \times c) \), since multiplication is a closed operation for integers \(-4 \times c\) is an integer and if we rewrite this as \( n = -4c\) then we have \(2 \times n\)
                \item \( 8c + 5 \) is odd because we can write this as follows \( 8c + 4 + 1 = 2 \times (4c + 2) + 1\), since multiplication and addition are closed operation for integers we know that, \(4c + 2\) is an integer, therefore we get number in a form of \(2n + 1\)
                \item \( (c^2 + 1) - (c^2 - 1) - 2\) equals 0, which is an even number.
            \end{itemize}

            \item Assume that \( m \) and \( n \) are particular integers:
            \begin{itemize}
                \item \( 6m + 8n \) is even as we can rewrite it as \( 2 \times (3m + 4n)\), and we already know that \(3m + 4n\) is an integer.
                \item \( 10mn + 7 \) is odd because \( 10mn + 6 + 1 = 2 \times (5mn + 3) + 1\) which is a number of the form \( 2n + 1\)
                \item If \( m > n > 0 \), \( m^2 - n^2 \) is composite:
                    \begin{itemize}
                        \item We factor \( m^2 - n^2 \) using the difference of squares to obtain \( (m + n)(m - n) \).
                        \item Since \( m > n \), both \( m + n \) and \( m - n \) are integers greater than 1.
                        \item Therefore, \( m^2 - n^2 = (m - n) \times (m + n)\) is a product of two integers greater than 1.
                        \item This implies that \( m^2 - n^2 \) has divisors other than 1 and itself, so it cannot be prime.
                        \item Hence, \( m^2 - n^2 \) is composite by definition.
                    \end{itemize}
            \end{itemize}
        \end{enumerate}
\end{document}