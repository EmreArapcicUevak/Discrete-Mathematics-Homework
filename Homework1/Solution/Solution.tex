\documentclass[a4paper, 10pt]{article}

\usepackage[margin = 1in]{geometry} % for spacing around
\usepackage{graphicx} % for including images in your pdfs
\usepackage{xcolor} % for including colors in your pdf
\usepackage{soul} % for text decoration
\usepackage[utf8]{inputenc} % for encoded text
\usepackage[T1]{fontenc}
\usepackage{setspace} % for setting different line spacings between paragrafs.
\usepackage{enumerate} % for letting us get more detailed enumerate lists
\usepackage{multirow} % to let us combine more rows together
\usepackage{colortbl} % for decorating tables
\usepackage{amsmath} % used for representing more complicated math displays
\usepackage{supertabular}
\usepackage{longtable} % both of these packages are used to making really big tables
\usepackage{wrapfig} % allows us to wrap text around figures
\usepackage{fancyhdr} % for making fancy headers
%\usepackage{bibtex} % for making better bibliographies
\usepackage[pdftex]{hyperref} % for letting us make links
\usepackage{lscape} % Allows us to flip from portrait to landspace
\usepackage{tikz} % for high detailed drawing
\usepackage{multicol} % To put things side by side
\usepackage{rotating} % For rotating objects
% \usepackage{draftwatermark} % For adding watermarks
\usepackage{MnSymbol} % for using multiple symbols
\usepackage{mathtools} % Used for more math symbols
\usepackage{xfrac} % For more complciated fractions and to add derivitives
\usepackage{hyperref} % for hyper links
\usepackage{enumitem} % for better enum lists
\usepackage{tcolorbox} % for adding colored text boxes
\usepackage{bm} % Adding bold text to math inputs
\usepackage{pgfplots} % Used for plotting functions
\usepackage{background}

% Settings for the background package
\backgroundsetup{
    scale=1, % Size - adjust as necessary
    color=black, % Color of the logo
    opacity=0.1, % Opacity - adjust to your liking; 0 = fully transparent, 1 = fully opaque
    angle=0, % Rotation angle if you want to rotate the logo
    position=current page.center, % Position - you can change this as necessary
    vshift=0cm, % Vertical shift - adjust as necessary
    hshift=0cm, % Horizontal shift - adjust as necessary
    contents={\includegraphics[width=10cm]{../../Logo}} % Replace 'your_logo_filename' with the name of your logo file
}

% Setting up the default image path
\graphicspath{{./Images/}}

% Implementing authro details
\title{Discrete Mathemathics Homework}
\author{Emre Arapcic-Uevak}
\date{}

% Setting up the fancy page style
\fancypagestyle{customStyle}{
	\lhead{} \chead{} \rhead{}
	\lfoot{} \cfoot{\thepage} \rfoot{}
	\renewcommand{\headrulewidth}{0pt}
	\renewcommand{\footrulewidth}{1pt}
}
\pagestyle{customStyle}

% Setting up hyperref options
\hypersetup {
	colorlinks = false,
	citecolor = black,
	filecolor = blue,
	linkcolor = blue,
	urlcolor = blue,
	pdftex
}

% Custom commands


\begin{document}
	\maketitle
	\vspace{5mm}
	
	\begin{abstract}
		\begin{center}
			\noindent This document encompasses a detailed set of solutions for discrete mathematics homework assignments. Covering core topics such as set theory, logic, relations, and graph theory, each solution is meticulously worked out to ensure clarity and accuracy. The problems addressed in this compilation have been selected based on their relevance and the challenges they pose, offering students a comprehensive resource to validate their work and deepen their understanding of the subject. By presenting both the problem statements and their corresponding solutions, this collection aims to be a valuable reference for students navigating the intricacies of discrete mathematics.
		\end{center}
	\end{abstract}
	\pagebreak
	
	\tableofcontents
	\pagebreak

    \section{Problem 1}

    \begin{enumerate}
        \item Some real numbers, when added together, are less than when subtracted from each other. \\
        \textbf{Answer}: False.

        \item There's a real number whose square is smaller than itself. \\
        \textbf{Answer}: True for numbers in the interval (0, 1).

        \item For any positive whole number, its square is at least as big as the number itself. \\
        \textbf{Answer}: True.

        \item When you add any two real numbers, their sum is always less than or equal to the sum of their absolute values. \\
        \textbf{Answer}: True.
    \end{enumerate}

    \section{Problem 2}

    \begin{enumerate}
        \item Is \( B \) a subset of \( A \)? \\
        \textbf{Answer}: No, because \( j \) is in \( B \) but not in \( A \).

        \item Is \( C \) a subset of \( A \)? \\
        \textbf{Answer}: Yes.

        \item Is \( C \) a subset of \( C \)? \\
        \textbf{Answer}: Yes, any set is a subset of itself.

        \item Is \( C \) a proper subset of \( A \)? \\
        \textbf{Answer}: Yes, because \( C \) is a subset of \( A \) and \( C \) is not equal to \( A \).
    \end{enumerate}


    \section{Problem 3}

        \subsection{Problem 3.1}
            Write the domain and co-domain of \( G \). \\
            \textbf{Answer:} \\
            Domain of \( G \): \( C = \{1, 2, 3, 4\} \) \\
            Co-domain of \( G \): \( \{c\} \)

        \subsection{Problem 3.2}
            Find \( G(1) \), \( G(2) \), \( G(3) \), and \( G(4) \). \\

            \textbf{Answer:} \\
            \begin{align*}
                G(1) &= c \\
                G(2) &= c \\
                G(3) &= c \\
                G(4) &= c \\
            \end{align*}

\end{document}
